\chapter{Abstract}
Modern technologies has allowed for the opportunity to collect personal data about human activities related to health. Some activities such as heart rate and physical activity can be collected passively and fully automated, but collecting psychological activities requires the user to manually log the data. Traditionally this has been achieved with pen and paper. While modern solution utilizing smartphones and smartwatches has greatly improved this, there still is a need to reduce the burden related to self tracking. Therefore this thesis proposes a solution using a wristband device together with arm gestures for self tracking. This will allow for tracking psychological activities in the moment, and allowing the user to rate the severeness on a continues scale similar to a Visual Analogue Scale (VAS). The wristband should be used together with a companion app that can import, display and export the data collected on the wristband. The thesis details the design and implementation of this solution. An experiment was designed and conducted to test the implemented solution compared to digital version of a VAS. In the experiment participants rated shades of grey and numbers using both the wristband with two different gestures and a digital VAS.

The results showed that the accuracy of the solution was not significantly different from the digital VAS, when rating shades of grey. When rating numbers the digital VAS performed significantly better. Mapping the results to a 0-10 scale (resembling a VAS), the mean absolute error using the wristband was within $\pm$1, about 50\% of the data points were within $\pm$1 and about 80\% of the data points were within $\pm$2, concluding that the solution is reliable. Comparing data entry time showed no significant difference between the wristband and digital VAS.