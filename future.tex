\chapter{Future Work}\label{future_ch}

Todo: introduction to chapter.. 

\section{Improvement to the Experiment}
The experiment conducted was a success, but if anyone was to conduct the experiment again a few improvements could be made. With the design of the slider and submit button the button is positioned in the center below the slider. This had the unintended effect of serving as a reference point for the middle of the slider. Simply changing the position of the submit button to either side of the slider would resolve this. As discussed in the results, we humans are quite good at finding the center of a slider anyway, so this is smaller detail that most likely won't have a significant impact on the outcome.

It would be interesting to see if the results of the experiment is representative of a more general population, so recruiting participants with a broader demographic would give an idea of this. Likewise it would be interesting to see if the observed results will stay the same with a larger data set using more participants. In would also be optimal to increase the number of stimuli for each exercise, although it is a trade off since it could lead to fatigue.

Conducting another experiment where only 11 shades of grey and numbers between 0 and 10 would be of interest, to see how the three input methods would compare in a setting closer to the real life use case. Likewise an experiment without a ground truth could be of interest to compare the three input methods, e.g. where participants are asked to rate pain levels, mood or other psychological events. Such an experiment might be harder to conduct since you would need to provide a stimuli for the psychological events, for pain that could be easy, but might not be ethical. It is also harder to compare the results since there won't exist a ground truth, in this case it would be interesting to see how similar the results would be. Such an experiment could be conducted over a longer period of time, where the participants are asked to use both the digital VAS and one or both of the gestures to rate psychological events.

\section{Case Study}
To further develop the work of this thesis, the next step would be to conduct a case study. This could be done similar to the work of Larsen et al.\cite{eg}, where a patient suffering from PTSD was asked to log whenever he experienced a specific event related to his illness using the smartbutton. The difference here would be that the patient would be asked to rate whenever he/she using the wristband and one of the two gestures (arm or wrist). PTSD is just one example that could be used in a case study, it would be equally interesting to conduct a case study on a patient suffering from migraines, chronic pain, anxiety, panic attacks or any similar condition where the common thing is that it will create value to track whenever the events of their conditions occurred and how severe the events are.

With such a case study it would be interesting to see if the tracking of severeness provides additional value over just tracking when events occur. It would be interesting to investigate the compliance of using the wristband with the hand or wrist gestures compared to the simpler smartbutton.

\section{Improvements to Companion App}
The companion app designed and implemented in this thesis was developed with the intentions to be a MVP, therefor some features where left out, but given more time it would create value to improve the companion app. Visualization of the data collected could make it easier to spot patterns. A simple proposal would be bar plot with days along x-axis and the number of tracked events on the y-axis, the bars could be colored to represent the severeness logged. If a given day had 5 data points with severeness rated at 1, 3, 6, 8 and 9 the bar could be colored with multiple color bands based on a color scale going from green, to yellow to red (like in the Mood tracker \ref{mood}). This is just one suggestion, there exist many papers on how to visualize such data, it is important that the visualization is thoroughly tested on a wide demographic to ensure that it can be understood by the users. Beside a better visual presentation of the collected data, it would be interesting to the location of the observation, while this isn't possible with the wristband device, this can be achieved by synchronizing the time stamps of the logged data to the smartphone location history. This could provide additional value to the treatment process. 

While these are the largest improvement that can be made to the companion some other minor adjustments could be made. In order to use companion app the wristband has to be connected first, this is of course important to program the wristband and import the data from it, but that means the user can't aces the collected data without first connecting the wristband. This is of course a inconvenience, the app should let the user see the collected data and only require the wristband to be connected when importing data or programming the wristband. Improvements could be made to the UI of the app to make it more visually pleasing, but that hasn't been the focus of this thesis. Features such as deleting individual data points would also improve the user experience of the app. 

\section{Investigating the Wrist's Critical Point}
In connection to conducting a case study and improving the companion app it would be reasonable to investigate the conditions surrounding the \say{critical point} of the wrist gesture. As mentioned in the  Chapter Results \& Discussions\ref{res_and_dis}, the \say{critical point} is the condition observed that caused the participant's responses to over estimated both the shades of grey and numbers when they where on the lower half of the scale, and under estimate the responses for the upper half of the scale. As mentioned this is likely do the bone structure of the lower arm causing the movement of the gesture to be non linear. It would be interesting to investigate this further and try to develop a mathematical model to compensate for this. Another approach could investigate the use of machine learning and see if that could compensate for this condition, all though for this to be successful a larger data set would most likely be required.


