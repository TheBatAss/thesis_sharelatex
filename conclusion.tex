\chapter{Conclusion}\label{con_ch}
In this thesis the design and implementation of a wristband device and companion app for manual collection of data on subjective experiences using a gesture based system. An experiment was conducted in order to compare the performance of the wristband with a digital version of a Visual Analogue Scale.

The thesis analyzed the problem and the key concepts that would form the foundation for the proposed solution. The two gestures of rotating the wrist and raising the arm was chosen since the corresponds to the roll and pitch axis, the yaw axis was discarded since it requires a reference point, and thus couldn't be used is the same way as the other two gestures.

The design of and implementation of the solution was described in detail. The wristband device would have a companion app to collect, display and export the data from the wristband. First a proof of concept was developed using of the shelf components, which was a success and laid the groundwork for the prototype. For said prototype the MetaMotionR from mbientlab was used as the wristband device and an Android app was developed to go along with it. The prototype worked and was able to log values on a continuous 0-10 scale based on the gestures together with a time stamp. The data could be imported to the companion app which would displayed the results and enabled the user to export it to a CSV file via email. 

In order to validate the performance of the solution an experiment was designed and conducted. Participants where tasked with rating different shades of grey and integers (stimuli), using both of the gesture based methods and digital VAS (input methods). The experiment consisted of six exercises (combinations of the two stimuli and three input methods), where each exercise consisted of 20 trials, where each trial had a random stimuli. Twenty four participants completed the experiment for a total of 2880 data points collected.

The collected data was analyzed for performance in accuracy and response time. The results showed that both of the designed input methods (arm and wrist gestures) had no significant difference in accuracy to the digital VAS when rating shades of grey. When rating numbers the digital VAS showed to be much more accurate, but when mapping the results to a 0-10 scale, both of the gesture based inputs had e mean absolute within $\pm$1, and about 50\% of the results was within this range and about 80\% within $\pm$2. Thus it must be concluded that both methods are reliable as a replacement for a VAS. When comparing response times there was no significant difference between the methods. Based on user feedback the wrist gesture was a little more preferred and they felt it to be more comfortable, although they felt the arm gesture to be more accurate. Since both gestures performed equally well and are equally preferred, it would be recommended to use the wrist gesture since it provides more comfort.

Lastly improvements where suggested that could improve the companion app and the experiment. A few points of interests was described in case the experiment was to be replicated, and it was suggested to do a case study in order to investigate the real life performance of the solution.