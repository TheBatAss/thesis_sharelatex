\chapter{Conclusion}\label{con_ch}
This thesis cowered the design and implementation of a wristband device and companion app for manual collection of data on subjective experiences using a gesture based system. An experiment was conducted in order to compare the performance of the wristband with a touch screen implementation of a Visual Analogue Scale (VAS). The results showed the designed solution to be feasible.

The thesis analyzed the problem and the key concepts that would form the foundation for the proposed solution. The two gestures of rotating the wrist and raising the arm was chosen since they corresponds to the roll and pitch axis.

The design of and implementation of the solution was described in detail. The wristband device would have a companion app to collect, display and export the data from the wristband. First a proof of concept was developed using off the shelf components, which was a success and laid the groundwork for a prototype. The prototype worked and was able to log values on a continuous 0-10 scale based on the gestures together with a time stamp.

In order to validate the performance of the solution an experiment was designed and conducted. Participants where tasked to rate different shades of grey and integers (stimuli), using both the gesture based methods and touch based VAS input methods. The collected data was analyzed for performance in accuracy and task completion time. The results showed that both of the designed input methods (arm and wrist gestures) had no significant difference in accuracy to the touch based VAS when rating shades of grey. When rating numbers the touch based VAS showed to have higher accuracy, but when mapping the results to a 0-10 scale, both of the gesture based inputs had e mean absolute within $\pm$1, and about 50\% of the results was within this range and about 80\% within $\pm$2. Thus it must be concluded that both methods are reliable as a replacement for a VAS. When comparing task completion times there was no significant difference between the methods. Based on user feedback the wrist gesture was a little more preferred and they felt it to be more comfortable, although they perceived the arm gesture to be more accurate. Since both gestures performed equally well and are equally preferred, it would be recommended to use the wrist gesture since it provide more comfort.

Finally, improvements where suggested that could improve the companion app and the experiment. In case the experiment was to be replicated it would be of interest to see if the results applies to a more general population and if the result will stay the same with a larger data set. A case study was suggested in order to investigate the real life performance of the solution.